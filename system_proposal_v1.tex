
\documentclass[a4j,titlepage]{ujarticle}
\usepackage[dvipdfmx]{graphicx}
\usepackage{url}
\usepackage{listings}


\title{
{なんとかシステム
\\
システム提案書}
\author{\\
\\
\\
\\
\\
Outing Corporation}
\date{\today}
}

\begin{document}
\maketitle


\tableofcontents



\clearpage


\section{はじめに}


\section{解決できる経営課題}
高知工科大学で開講されている講義である情報学群実験では、講義時間以降も課題の提出が終わらない現状があります。
具体的には、

(1) 提出のチェックが挙手制でありTeaching Assistant(以下TA)にとってわかりづらい。

(2) 挙手のタイミングが被り、同時にチェックすることができない。

(3) 課題提出後の質問に関しても同様に挙手制であり、課題のチェックとの区別ができない。

(4) 挙手に抵抗のある学生がいる。\\
という背景があり、それによりTAの残業が深刻なものになっています。TAには時給が発生しており、学生の課題提出が終わらないことにより、
学校側がTAに支払う金額も大きくなると考えられます。



 %こういった状況に対して学生側は授業中に課題の提出を終わらせることができず



\section{課題解決のための提案}
本提案書では前項で述べた課題を解決するものとして「システムの名前」を提案いたします。

(1)学生からの質問や課題チェックをまとめることで、TAがスムーズに課題のチェックができる状況を提供します。

(2)質問の保存のやつ

(3)複数ある課題の進捗状況を一覧でわかるようにすることで、TAが優先順位を考えて行動することができる状況を提供します。


\section{課題解決のための方法}
前項で説明した提案につきまして、具体的な方法を説明いたします。

(1)各ユーザーが自分の状況をを入力することで、TAが確認しやすいUIの作成

(2)ユーザーからの質問をデータベースに保存し、来年度それを参考することで ・・・ これはなんの課題なのか

(3)各班の進捗状況を入力し、TAがそれを監視することで、全ての班が平均的に課題を終わらせることができる状況を作る

\section{機能概要,前提条件,制約事項}

\subsection{機能概要}

(1)ユーザー登録をする

(2)管理者側は班数や問題数の設定をする

(3)利用者はログインし、進行状況やチェック数、質問とチェックの区別をする

(4)TAは進行状況をリアルタイムで知ることができる
\subsection{前提条件}

(1)Raspberry-piがネットワークに接続できる環境があること


\subsection{制約事項}

(1)管理者は課題の情報を入力することが必要である

\section{サービス利用までの流れ}
\subsection{人の流れ}
このシステムの利用者は、管理者側と実験を行う学生側の二者となります。
管理者は (仮)ネットワークに接続されているスマートフォンやパソコン等の端末からアプリケーションの管理者画面にログインします。
そして、講義の都度、その講義の課題について設定を行います。
学生は、班で一人がそのアプリケーションのログイン画面からログインし、課題を行う中で入力していきます。
システムの運用・保守については管理者が行い、質問等のデータベースのレコード等の編集も行うことができます。
障害が発生した場合は、再起動を行うことで前回更新した状況まで戻すことで対応します。


\subsection{データの流れ}
このシステムは、ログイン画面によって管理者側と利用者側の区別が行われます。そのため、システム自体は同様のものとなり、それとは別にサーバが
存在することでこのシステムの構成要素となります。
システム内部での情報の流れは以下の図1のようになっています。

サーバには利用者が登録の際に入力した個人情報や、質問の情報、課題の進捗状況などが格納されています。管理者は終了した講義の情報や次回の講義の
課題の情報を管理でき、利用者が前提条件に当てはまらなくなった場合には、登録の削除を行うことができます。
また、質問の一覧のデータを確認、削除することができ、そのデータベースの管理は管理者画面から行うことができます。

\section{想定する利用者}
このシステムが想定する利用者は、高知工科大学の教授、TA、学生となっています。

\section{導入・移行計画}
2018年2月1日をもってアプリケーションの公開を完了します。
\section{システムのハードウェア構成・ソフトウェア構成}
システムのハードウェア構成は、メインサーバとしての役割をはたすRaspberry-piが1台とし、そのシステムを管理する端末が1台となっています。

\section{運用・保守}
提案システムの運用・保守については、マニュアルのもと、全て管理者が行います。
\section{作業標準}
システム開発にかかる作業標準は貴社ご指定のものを使用します。
\section{品質管理}
システム開発にかかる品質管理は貴社ご指定のものを使用します。
\section{工程計画}
工程計画は次の通りです。

外部設計完了:2017年11月27日

内部設計完了:2017年12月18日

開発完了:2018年1月25日

導入:2018年2月5日
\section{体制}
このシステムの開発は弊社システム部門部員兼取締役社長の沖を中心として、7名のエンジニアにより実施します。
\section{システム化にかかる費用とその効果}
\subsection{費用}
システム化にかかる費用は以下の通りです。

票を入れる


\subsection{効果}
システム化による効果の試算を以下に示します。
どれだけお金が浮くかを計算するよ

\section{本システム提案のアピールポイント}
汎用性の高いシステムとなっているため、他の会議等での利用も可能です。
質問の内容が蓄積され、次年度の講義の改善へ繋げることができます。
\section{用語の定義}
\newpage

\end{document}
