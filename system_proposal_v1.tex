
\documentclass[a4j,titlepage]{ujarticle}
\usepackage[dvipdfmx]{graphicx}
\usepackage{url}
\usepackage{listings}


\title{
{なんとかシステム
\\
システム提案書}
\author{\\
\\
\\
\\
\\
Outing Corporation}
\date{\today}
}

\begin{document}
\maketitle


\tableofcontents



\clearpage


\section{はじめに}


\section{解決できる経営課題}
高知工科大学で開講されている講義である情報学群実験では、講義時間以降も課題の提出が終わらない現状があります。
具体的には、

(1) 提出のチェックが挙手制でありTeaching Assistant(以下TA)にとってわかりづらい。

(2) 挙手のタイミングが被り、同時にチェックすることができない。

(3) 課題提出後の質問に関しても同様に挙手制であり、課題のチェックとの区別ができない。

(4) 挙手に抵抗のある学生がいる。\\
という背景があり、それによりTAの残業が深刻なものになっています。TAには時給が発生しており、学生の課題提出が終わらないことにより、
学校側がTAに支払う金額も大きくなると考えられます。



 %こういった状況に対して学生側は授業中に課題の提出を終わらせることができず



\section{課題解決のための提案}
本提案書では前項で述べた課題を解決するものとして「システムの名前」を提案いたします。

(1)学生からの質問や課題チェックをまとめることで、TAがスムーズに課題のチェックができる状況を提供します。

(2)質問の保存のやつ

(3)進捗状況のやつ


\section{課題解決のための方法}
前項で説明した提案につきまして、具体的な方法を説明いたします。

(1)

(2)

(3)上と対応させるやつ

\section{機能概要,前提条件,制約事項}

\subsection{機能概要}

(1)ユーザー登録をする

(2)管理者側は班数や問題数の設定をする

(3)利用者はログインし、進行状況やチェック数、質問とチェックの区別をする

(4)TAは進行状況をリアルタイムで知ることができる
\subsection{前提条件}

(1)Raspberry-piがネットワークに接続できる環境があること


\subsection{制約事項}

\section{サービス利用までの流れ}
\subsection{データの流れ}


\section{想定する利用者}
\section{導入・移行計画}
\section{システムのハードウェア構成・ソフトウェア構成}
\section{運用・保守}
\section{作業標準}
\section{品質管理}
\section{工程計画}
\section{体制}
\section{システム化にかかる費用とその効果}
\subsection{費用}
\subsection{効果}
\section{本システム提案のアピールポイント}
\section{用語の定義}
\newpage

\end{document}
