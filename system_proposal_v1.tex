
\documentclass[a4j,titlepage]{ujarticle}
\usepackage[dvipdfmx]{graphicx}
\usepackage{url}
\usepackage{listings}


\title{
{課題提出型講義支援システム
\\
システム提案書}
\author{\\
\\
\\
\\
\\
Outing Corporation}
\date{\today}
}

\begin{document}
\maketitle


\tableofcontents



\clearpage


\section{はじめに}
% 近年、情報システムの発展に伴い、リアルタイム提出(や、〜〜〜など)が可能となっている。その結果、課題提出型講義が大学の講義として増えているみたいなことを言いたい
課題提出型の講義の一例として、高知工科大学の情報学群実験では、学生は各グループに分かれて、それぞれが与えられた課題に取り掛かります。
% そして、課題をこなすと、Teaching Assistant(以下TA)に、課題のチェックをしてもらい、合格するとそのグループは課題を終えることができます。(全然良い文章書けなかった)
学生は、この課題チェックや質問を、挙手によって行うが、% 問題点述べて
その結果、夜遅く(←ちょっと曖昧)まで課題が終わらないことが多々あります。
このような現状に対して、課題チェックや質問を効率良く行うことができる、課題提出型講義支援システムをご提案いたします。

% 全然良い文章書けない〜〜〜

\section{解決できる経営課題}
高知工科大学で開講されている講義である情報学群実験では、講義時間以降も課題の提出が終わらない現状があります。 % 実験のこと知らない人に向けての説明をしたい
その背景として、
\begin{description}
\item[(1)]提出のチェックが挙手制でありTeaching Assistant(以下TA)にとってわかりづらい。
\item[(2)]挙手のタイミングが被り、同時にチェックすることができない。
\item[(3)]課題提出後の質問に関しても同様に挙手制であり、課題のチェックとの区別ができない。
\item[(4)]挙手に抵抗のある学生がいる。
\end{description}
が挙げられます。それによって、TAの残業が深刻なものになっています。また、TAには時給が発生しており、学生の課題提出が終わらないことにより、
大学側がTAに支払う金額も大きくなると考えられます。



 %こういった状況に対して学生側は授業中に課題の提出を終わらせることができず



\section{課題解決のための提案}
本提案書では前項で述べた課題を解決するものとして「実験支援システム(仮)」を提案いたします。

\begin{description}
\item[(1)]学生からの質問や課題チェックをまとめることで、TAがスムーズに課題のチェックができる状況を提供します。
\item[(2)]学生からの質問は、その解答とともに、データベースに蓄積します。
\item[(3)]各グループの課題の進捗状況を一覧でわかるようにすることで、TAが優先順位を考えて行動することができる状況を提供します。
\end{description}

\section{課題解決のための方法}
前項で説明した提案につきまして、具体的な方法を説明いたします。
\begin{description}
\item[(1)]各ユーザーが自分の状況をを入力することで、TAが確認しやすいUIの作成 % UIとは
\item[(2)]ユーザーからの質問をデータベースに保存し、来年度それを参考することで ・・・ これはなんの課題なのか
\item[(3)]各班の進捗状況を入力し、TAがそれを監視することで、全ての班が平均的に課題を終わらせることができる状況を作る
\end{description}
\section{機能概要,前提条件,制約事項}

\subsection{機能概要}
\begin{description}
\item[(1)]利用者はユーザー登録をします。
\item[(2)]管理者側はグループ数や課題の数を指定します。
\item[(3)]利用者はログインし、進捗状況やチェック数、質問とチェックの区別をする。 % ←分かりづらい
\item[(4)]TAは、利用者の課題の進捗状況をリアルタイムで知ることができます。
\end{description}

\subsection{前提条件}
\begin{description}
\item[(1)]Raspberry-pi 3がネットワークに接続できる環境で、このシステムを利用することが前提条件です。
\end{description}

\subsection{制約事項}
\begin{description}
\item[(1)]管理者は、利用者のグループ数や課題の数を、あらかじめ入力することが必要です。
\end{description}

\section{サービス利用までの流れ}
\subsection{人の流れ}
このシステムの利用者は、管理者側と実験を行う学生側の二者となります。 % ←の文章だと、利用者の中に管理者がいるが、他のところで、実験を行う学生のことを利用者と表現している箇所がある。
管理者は、講義が始まる前に、ネットワークに接続されているスマートフォンやパソコン等の端末から、システムの管理者画面にログインします。
そして、講義の課題について設定を行います。
学生側は、各グループの代表一人が、そのシステムにログインします。
そして、課題を行う中で入力していきます。 % 何を入力する?
システムの運用・保守については管理者が行い、質問等のデータベースのレコード等の編集も行うことができます。 % データベースのレコードが分からない
障害が発生した場合は、再起動を行うことで前回更新した状況まで戻すことで対応します。 % 例えばどんな障害?システム自体の再起動かアプリだけの再起動か?


\subsection{データの流れ}
このシステムは、ログイン画面によって管理者側と利用者側の区別が行われます。そのため、システム自体は同様のものとなり、それとは別にサーバが % 学生側を利用者という言い方は良くない。ログイン画面で区別とは?システム自体が同様を別の言い方で
存在することでこのシステムの構成要素となります。
システム内部での情報の流れは以下の図1のようになっています。 % どんな図作る?

サーバには利用者が登録の際に入力した個人情報や、質問の情報、課題の進捗状況などが格納されています。管理者は終了した講義の情報や次回の講義の % ここの利用者は学生側のこと
課題の情報を管理でき、利用者が前提条件に当てはまらなくなった場合には、登録の削除を行うことができます。
また、質問の一覧のデータを確認、削除することができ、そのデータベースの管理は管理者画面から行うことができます。

\section{想定する利用者}
このシステムが想定する利用者は、高知工科大学の教授、TA、学生となっています。 % もうちょっと詳しく。どんな学生かとか。利用者を別の言い方にしたい。

\section{導入・移行計画}
2018年2月1日をもってアプリケーションの公開を完了します。

\section{システムのハードウェア構成・ソフトウェア構成}
システムのハードウェア構成は、メインサーバとしての役割をはたすRaspberry-pi 3が1台、そのシステムを管理する端末が1台となっています。 % システムを管理する端末って何?ソフトウェア構成が書かれてないが、ソフトウェア構成とは何?

\section{運用・保守}
提案システムの運用・保守については、マニュアルのもと、全て管理者が行います。 % 管理者って誰?これまでに上手いこと説明できてたっけ?

\section{作業標準}
システム開発にかかる作業標準は貴社ご指定のものを使用します。

\section{品質管理}
システム開発にかかる品質管理は貴社ご指定のものを使用します。

\section{工程計画}
工程計画は次の通りです。

外部設計完了:2017年11月27日

内部設計完了:2017年12月18日

開発完了:2018年1月25日

導入:2018年2月5日

\section{体制}
このシステムの開発は弊社システム部門の沖を中心として、7名のエンジニアにより実施します。

\section{システム化にかかる費用とその効果}
\subsection{費用}
システム化にかかる費用は以下の通りです。

% 表どうしよう。誰か計算して~
\begin{table}[htb]
  \begin{tabular}{|c|c|c|c|c|} \hline
    項目 & 単価(円) & 数量 & 金額(円) & 備考 \\ \hline
     &  &  &  &  \\ \hline
     &  &  &  &  \\ \hline
     &  &  &  &  \\ \hline
    \multicolumn{3}{c||}{合計}& 1,000円 &  \\ \hline
  \end{tabular}
\end{table}


\subsection{効果}
システム化による効果の試算を以下に示します。
% どれだけお金が浮くかを計算するよ

\section{本システム提案のアピールポイント}
汎用性の高いシステムとなっているため、他の会議等での利用も可能です。
質問の内容が蓄積され、次年度の講義の改善へ繋げることができます。

\section{用語の定義}
\newpage

\end{document}
