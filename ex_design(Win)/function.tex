\section{機能説明}
% ここから本文を書いてください.
本システムにおいて、管理者は講義中に課題を行う学生の進捗の確認や、質問の確認・回答を行うことができます。
また、学生は進捗の報告や、質問を行うことができます。
この章では、本システムで利用できる機能について説明します。

 %要追加
\subsection{アカウント登録機能}
本システムを利用するために、アカウント登録を行う必要があります。
本システムには、あらかじめ管理者アカウントが一つ用意されています。
管理者のアカウントは、すでに登録されている管理者のアカウントからのみ登録することができます。
学生のアカウントは、管理者のアカウント登録画面またはログイン画面から登録することができます。
使用できる機能については以下の通りです。
\begin{enumerate}
  \item 管理者による他の管理者アカウントの登録
  \item 管理者による学生アカウントの登録
  \item 学生による学生アカウントの登録
\end{enumerate}
%このシステム → 本システム に直しました  沖&岡田 → 谷脇
%「登録」と「作成」について 岡田&沖 → 谷脇
%文章全体の校正 メンバー全員 → 谷脇

\subsection{進捗確認機能}
進捗確認機能は、Raspberry Pi 3に接続された管理者端末で利用することができます。
授業を作成し、開くことによって、学生がログインした時に表示される画面が確定します。
その画面に応じて学生側は情報を送信することができます。
使用できる機能は以下の通りです。
\begin{enumerate}
  \item グループまたは学生個人の進捗状況の確認
  \item 各学生・グループの状態の確認%状態?
  \item 質問回答画面への遷移
  \item 課題編集画面への遷移
  \item 授業の終了
\end{enumerate}

\subsection{質問機能}
質問機能は、Raspberry Pi 3に接続された全ての端末から利用することができます。
学生は、開かれている授業に対して質問することができます。
管理者は、その質問に対して回答することができます。
管理者および学生は、過去の授業の質問を閲覧することができます。
この質問画面において使用できる機能は以下の通りです。
\begin{enumerate}
  \item 学生からの質問の送信
  \item 管理者が質問に対して回答
  \item 授業中の質問及び回答の閲覧
  \item 過去の授業の質問及び回答の閲覧
\end{enumerate}
