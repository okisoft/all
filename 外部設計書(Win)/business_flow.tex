\section{業務の流れ}
% ここから本文を書いてください.
本システムは主に学生の進捗状況並びに質問を整理することで管理者である教師の方々が円滑に授業を行うための、教務支援システムである。

管理者はシステムが内蔵してあるラズベリーパイを、管理者用PCに接続することで利用可能となる。
学生側はラズベリーパイからのWi-Fiをスマートフォンか別のラズベリーパイで受信することによりログインする。
主に管理者は授業に合わせて学生側に映し出す画面を作成することによって、学生ごとに課題をどの程度進めているのかを確認できる。
これは学生側から講義中に行なっている課題の進捗状況をリアルタイムで送信することで実現する。
また学生側は、講義内容や課題に対しての質問を送信する。
この入力された質問は管理者の画面上でどの学生による質問であるのかが表示される。
その場で回答を行なった場合でも、このシステムに書き込んだ場合でも質問自体は授業中であれば学生側も閲覧可能となっている。
この質問と、回答についてはDB(データベース)に蓄積されることにより、翌年度以降の講義改善に役立てることができる。
DBに蓄積されるデータは実際に利用された画面などのデータも格納しているため、一度設定を行うことで次回以降も利用可能になっている。
