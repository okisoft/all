\section{機能説明}
% ここから本文を書いてください.
本システムの概要として進捗確認機能、質問機能について説明します。
 %要追加
\subsection{進捗確認機能}
進捗確認機能は、Raspberry Pi 3に接続された管理者端末で利用することができます。
授業を作成し開くことによって、学生がログインした時に表示される画面が確定します。
その画面に応じて学生側は情報を送信することができます。
使用できる機能は以下の通りです。
\begin{enumerate}
  \item グループまたは学生個人の進捗状況の確認
  \item 各学生・グループの状態の確認%状態?
  \item 質問回答画面への遷移
  \item 課題編集画面への遷移
  \item 授業の終了
\end{enumerate}

\subsection{質問機能}
授業に対して設置された質問画面において、この機能を利用することができます。
この質問機能において使用できる機能は以下の通りです。
\begin{enumerate}
  \item 学生からの質問の送信
  \item 管理者が質問に対して回答
  \item 授業中の質問及び回答の閲覧
  \item 過去の授業の質問及び回答の閲覧
\end{enumerate}
