\section{機能説明}
% ここから本文を書いてください.
本システムの概要として、管理者画面について、また、主な機能として画面作成機能、質問機能について説明します。
 %要追加
\subsection{管理者画面}
管理者画面では、以下の機能を使用できます。
\begin{enumerate}
  \item 管理者・ユーザの管理
  \item データベースの管理%講義や質問のデータベース
  \item 画面作成
  \item 画面の提供・提供中画面の確認
\end{enumerate}

\subsection{画面作成機能}
画面作成機能は、Raspberry Pi 3に接続された管理者端末で利用することができます。
作成された画面は、学生側に提供することができ、その画面に応じて学生側は情報を
送信することができます。
使用できる機能は以下の通りです。
\begin{enumerate}
  \item 進捗確認画面の追加・削除
  \item 質問画面の追加・削除
  \item グループ作成画面の追加・削除(みたいな書き方で複数追加予定)
\end{enumerate}

この画面作成機能によって作成された画面を、学生側が閲覧できるように公開することができます。

\subsection{質問機能}
画面作成機能によって作成された質問画面において、この機能を利用することができます。(他にも画面作成で
作れる画面が増えれば、このように記述することで同一性を持たせることができると思いました。)
この質問機能において使用できる機能は以下の通りです。
\begin{enumerate}
  \item 質問入力枠の追加・削除
  \item 送信ボタンの追加・削除
  \item 質問及び回答の閲覧機能の追加・削除(他にもあれば追加)
\end{enumerate}
