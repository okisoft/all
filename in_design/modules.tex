\section{ルーティングとMVC}
\subsection{ルーティング}
Railsの規約に従ったurl規則を以下の表に定義します。
呼び出されたurlとHTTPメソッドによってRailsで呼び出すcontrollerのアクションを定義します。

\begin{table}[]
\centering
\caption{ルーティング一覧}
\label{my-label}
\begin{tabular}{lll}
\hline
\multicolumn{1}{|l|}{No.}                          & \multicolumn{1}{l|}{METHOD} & \multicolumn{1}{l|}{URL}                                                                                                                                    \\ \hline
\multicolumn{1}{|l|}{Controller\#Action}           & 画面名(仮定義)                    &                                                                                                                                                             \\ \cline{1-1}
---                                                & ------                      & ---                                                                                                                                                         \\
-----------                                        &                             &                                                                                                                                                             \\ \hline
\multicolumn{1}{|l|}{1}                            & \multicolumn{1}{l|}{GET}    & \multicolumn{1}{l|}{/login}                                                                                                                                 \\ \hline
\multicolumn{1}{|l|}{sessions\#new}                & ログイン画面                      &                                                                                                                                                             \\ \cline{1-2}
\multicolumn{1}{|l|}{2}                            & \multicolumn{1}{l|}{POST}   &                                                                                                                                                             \\ \cline{1-2}
\multicolumn{1}{|l|}{sessions\#create}             &                             &                                                                                                                                                             \\ \hline
\multicolumn{1}{|l|}{2}                            & \multicolumn{1}{l|}{DELETE} & \multicolumn{1}{l|}{/logout}                                                                                                                                \\ \hline
\multicolumn{1}{|l|}{sessions\#destroy}            &                             &                                                                                                                                                             \\ \hline
\multicolumn{1}{|l|}{3}                            & \multicolumn{1}{l|}{GET}    & \multicolumn{1}{l|}{/signup}                                                                                                                                \\ \hline
\multicolumn{1}{|l|}{users\#new}                   & 学生アカウント新規登録画面               &                                                                                                                                                             \\ \hline
\multicolumn{1}{|l|}{4}                            & \multicolumn{1}{l|}{POST}   & \multicolumn{1}{l|}{/users}                                                                                                                                 \\ \hline
\multicolumn{1}{|l|}{users\#create}                &                             &                                                                                                                                                             \\ \hline
\multicolumn{1}{|l|}{4}                            & \multicolumn{1}{l|}{PATCH}  & \multicolumn{1}{l|}{/users/:id}                                                                                                                             \\ \hline
\multicolumn{1}{|l|}{users\#update}                &                             &                                                                                                                                                             \\ \hline
\multicolumn{1}{|l|}{5}                            & \multicolumn{1}{l|}{GET}    & \multicolumn{1}{l|}{/users/:id/edit}                                                                                                                        \\ \hline
\multicolumn{1}{|l|}{users\#edit}                  & 登録情報編集画面                    &                                                                                                                                                             \\ \hline
\multicolumn{1}{|l|}{5}                            & \multicolumn{1}{l|}{POST}   & \multicolumn{1}{l|}{/students}                                                                                                                              \\ \hline
\multicolumn{1}{|l|}{students\#create}             &                             &                                                                                                                                                             \\ \hline
\multicolumn{1}{|l|}{5}                            & \multicolumn{1}{l|}{GET}    & \multicolumn{1}{l|}{/students/new}                                                                                                                          \\ \hline
\multicolumn{1}{|l|}{students\#new}                & 履修確認画面                      &                                                                                                                                                             \\ \hline
\multicolumn{1}{|l|}{6}                            & \multicolumn{1}{l|}{GET}    & \multicolumn{1}{l|}{/}                                                                                                                                      \\ \hline
\multicolumn{1}{|l|}{questions\#index}             & 開講中の授業画面                    &                                                                                                                                                             \\ \hline
\multicolumn{1}{|l|}{6}                            & \multicolumn{1}{l|}{GET}    & \multicolumn{1}{l|}{/home}                                                                                                                                  \\ \hline
\multicolumn{1}{|l|}{questions\#index}             & 開講中の授業画面                    &                                                                                                                                                             \\ \hline
\multicolumn{1}{|l|}{7}                            & \multicolumn{1}{l|}{GET}    & \multicolumn{1}{l|}{/questions/new}                                                                                                                         \\ \hline
\multicolumn{1}{|l|}{questions\#new}               & 質問画面                        &                                                                                                                                                             \\ \hline
\multicolumn{1}{|l|}{7}                            & \multicolumn{1}{l|}{POST}   & \multicolumn{1}{l|}{/questions/create}                                                                                                                      \\ \hline
\multicolumn{1}{|l|}{questions\#create}            & 質問画面                        &                                                                                                                                                             \\ \hline
\multicolumn{1}{|l|}{8}                            & \multicolumn{1}{l|}{PATCH}  & \multicolumn{1}{l|}{/progress}                                                                                                                              \\ \hline
\multicolumn{1}{|l|}{progress\#update}             &                             &                                                                                                                                                             \\ \hline
\multicolumn{1}{|l|}{8}                            & \multicolumn{1}{l|}{POST}   & \multicolumn{1}{l|}{/achievements}                                                                                                                          \\ \hline
\multicolumn{1}{|l|}{achievements\#create}         &                             &                                                                                                                                                             \\ \hline
\multicolumn{1}{|l|}{9}                            & \multicolumn{1}{l|}{GET}    & \multicolumn{1}{l|}{/years}                                                                                                                                 \\ \hline
\multicolumn{1}{|l|}{years\#index}                 & 年度選択画面                      &

\end{tabular}
\end{table}

\begin{table}[]
\centering
\begin{tabular}{lll}
\hline
\multicolumn{1}{|l|}{10}                           & \multicolumn{1}{l|}{GET}    & \multicolumn{1}{l|}{/years/:id/lecture\_times}                                                                                                              \\ \hline
\multicolumn{1}{|l|}{lecture\_times\#index}        & 授業回選択画面                     &                                                                                                                                                             \\ \hline
\multicolumn{1}{|l|}{11}                           & \multicolumn{1}{l|}{GET}    & \multicolumn{1}{l|}{\begin{tabular}[c]{@{}l@{}}/years/:year\_id/lecture\_times/\\ :lecture\_time\_id/questions\end{tabular}}                                \\ \hline
\multicolumn{1}{|l|}{questions\#post\_index}       & 過去の質問画面                     &                                                                                                                                                             \\ \hline
\multicolumn{1}{|l|}{12}                           & \multicolumn{1}{l|}{GET}    & \multicolumn{1}{l|}{/groups}                                                                                                                                \\ \hline
\multicolumn{1}{|l|}{groups\#index}                & グループ選択画面                    &                                                                                                                                                             \\ \cline{1-2}
\multicolumn{1}{|l|}{12}                           & \multicolumn{1}{l|}{POST}   &                                                                                                                                                             \\ \cline{1-2}
\multicolumn{1}{|l|}{groups\#create}               &                             &                                                                                                                                                             \\ \cline{1-2}
\multicolumn{1}{|l|}{12}                           & \multicolumn{1}{l|}{PATCH}  &                                                                                                                                                             \\ \cline{1-2}
\multicolumn{1}{|l|}{groups\#update}               &                             &                                                                                                                                                             \\ \hline
\multicolumn{1}{|l|}{13}                           & \multicolumn{1}{l|}{POST}   & \multicolumn{1}{l|}{/group\_members}                                                                                                                        \\ \hline
\multicolumn{1}{|l|}{group\_members\#create}       &                             &                                                                                                                                                             \\ \hline
\multicolumn{1}{|l|}{14}                           & \multicolumn{1}{l|}{GET}    & \multicolumn{1}{l|}{/admin}                                                                                                                                 \\ \hline
\multicolumn{1}{|l|}{admin/lectures\#index}        & 管理者用ホーム画面                   &                                                                                                                                                             \\ \hline
\multicolumn{1}{|l|}{15}                           & \multicolumn{1}{l|}{GET}    & \multicolumn{1}{l|}{/admin/lectures/:id/lecture\_times}                                                                                                     \\ \hline
\multicolumn{1}{|l|}{admin/lecture\_times\#index}  & 授業回選択画面                     &                                                                                                                                                             \\ \hline
\multicolumn{1}{|l|}{16}                           & \multicolumn{1}{l|}{GET}    & \multicolumn{1}{l|}{\begin{tabular}[c]{@{}l@{}}/admin/lectures/:lecture\_id/\\ lecture\_times/:lecture\_time\_id/\\ questions\end{tabular}}                 \\ \hline
\multicolumn{1}{|l|}{admin/questions\#index}       & 回答画面                        &                                                                                                                                                             \\ \hline
\multicolumn{1}{|l|}{17}                           & \multicolumn{1}{l|}{GET}    & \multicolumn{1}{l|}{\begin{tabular}[c]{@{}l@{}}/admin/lectures/:lecture\_id/\\ lecture\_times/:lecture\_time\_id/\\ progress\end{tabular}}                  \\ \hline
\multicolumn{1}{|l|}{admin/progress\#index}        & 進捗確認画面                      &                                                                                                                                                             \\ \hline
\multicolumn{1}{|l|}{18}                           & \multicolumn{1}{l|}{GET}    & \multicolumn{1}{l|}{/admin/lectures/:id/years}                                                                                                              \\ \hline
\multicolumn{1}{|l|}{admin/years\#index}           & 質問閲覧選択画面1                   &                                                                                                                                                             \\ \hline
\multicolumn{1}{|l|}{19}                           & \multicolumn{1}{l|}{GET}    & \multicolumn{1}{l|}{\begin{tabular}[c]{@{}l@{}}/admin/lectures/:lecture\_id/\\ years/:year\_id/lecture\_times\end{tabular}}                                 \\ \hline
\multicolumn{1}{|l|}{admin/lecture\_times\#index}  & 質問閲覧選択画面2                   &                                                                                                                                                             \\ \hline
\multicolumn{1}{|l|}{20}                           & \multicolumn{1}{l|}{GET}    & \multicolumn{1}{l|}{\begin{tabular}[c]{@{}l@{}}/admin/lectures/:lecture\_id/\\ years/:year\_id/lecture\_times/\\ :lecture\_time\_id/questions\end{tabular}} \\ \hline
\multicolumn{1}{|l|}{admin/questions\#past\_index} & 質問閲覧選択画面3                   &                                                                                                                                                             \\ \cline{1-1}

\end{tabular}
\end{table}

\newpage

\subsection{View層}
\begin{enumerate}


\item  sessions/new.html \\
名称:ログイン画面 \\
概要:ログインをする\\
処理:\\
- ログインを押すと,入力されたユーザIDとパスワードを情報として/loginにPOSTリクエストでルーティングにリクエストする.\\
- 新規登録を押すと,/signupにGETメソッドでルーティングにリクエストする.\\

\item  users/new.html \\
名称:アカウント作成画面 \\
概要:アカウントを作成する \\
処理:\\
- 登録を押すと,入力されたユーザID,氏名,パスワード,確認用パスワードを情報として,/signupにPOSTメソッドでルーティングにリクエストする.

\item  users/edit.html \\
名称:アカウント編集画面 \\
概要:アカウントを編集する \\
処理:\\
- 変更を押すと,入力されたユーザID,氏名,旧パスワード,新パスワード,確認用新パスワードを情報として,/usersにPATCHメソッドでルーティングにリクエストする.

\item  students/new.html \\
名称:履修登録画面 \\
概要:履修の登録を行う \\
処理:\\
- 履修するを押すと,ログインしているユーザを情報として/studentsにPOSTメソッドでルーティングにリクエストする.

\item  groups/index.html\\
名称:グループ選択画面 \\
概要:グループの一覧を表示する \\
処理:\\
- 参加を押すとログインしているユーザを情報として,/group\_membersにPOSTメソッドでルーティングにリクエストする.\\
- 新規グループ作成を押すと,入力されたグループ名を情報として/groupsにPOSTメソッドでルーティングにリクエストする.\\
- 過去の質問を押すと/yearsにGETメソッドでルーティングにリクエストする.

\item  questions/index.html \\
名称:学生用ホーム画面 \\
概要:質問の確認や進捗状況の送信を行う \\
処理:\\
- 質問をするを押すと,/questions/newにGETメソッドでルーティングにリクエストする.

\item  questions/new.html \\
名称:質問画面 \\
概要:課題の質問を行う \\
処理:\\
- 質問または緊急を押すと,入力された質問を情報として,/questions/createにPOSTメソッドでルーティングにリクエストする.

\item  years/index.html \\
名称:年度選択画面 \\
概要:過去の授業の年度一覧を表示する \\
処理:\\
- 一覧で表示された年度のボタンを押すと,入力されたボタンを情報として,/years/:id/lecture\_timesにGETメソッドでルーティングにリクエストする.

\item  lecture\_times/index.html \\
名称:開講回選択画面 \\
概要:過去の授業回一覧を表示する \\
処理:\\
- 一覧で表示された授業回のボタンを押すと,入力されたボタンを情報として,/years/:id/lecture\_times/:id/questionsにGETメソッドでルーティングにリクエストする.


\item  /static\_page/not\_open.html \\
名称:未開講時画面 \\
概要:未開講時であることを表示する \\
処理:\\
---

\item  /shared/progress.html \\
名称:進捗情報入力画面 \\
概要:進捗状況を入力する.この画面は学生用ホーム画面,年度選択画面,回公開選択画面で表示される.\\
処理:\\
- 更新を押すと,入力された課題のチェックボックスを情報として,/progressにPATCHメソッドでルーティングにリクエストする.\\
- 確認を押すと,入力された情報を/achievmentsにPOSTメソッドでルーティングにリクエストする.

\end{enumerate}

\newpage

\subsection{Controller層}

\begin{enumerate}

\item  sessions\_controller.rb\\
名称:セッション情報処理  \\
概要:ユーザのセッション情報を処理する \\
処理: \\
- new:sessions/new.htmlを表示させる. \\
- create:ユーザIDとパスワードで認証を行い,認証されたユーザのセッションを作成してユーザによって以下のルーティングにリクエストをする.\\
  ユーザが教員:/adminをGETメソッドでルーティングにリクエストする.\\
  ユーザが学生:\\
  -開講していなければ/static\_page/not\_open.htmlを表示させる.\\
  -開講していたが,履修していなければ,/students/newにGETメソッドでルーティングにリクエストする.\\
  -開講していて,履修しているならば/homeにGETメソッドでルーティングにリクエストする.\\
  -認証が失敗したならばsessions/new.htmlを表示させる.

\item  users\_controller.rb\\
名称:ユーザ情報処理\\
概要:ユーザ情報を処理する\\
処理:\\
- new:users/new.htmlを表示させる.\\
- create:入力された情報が正しい情報かを判断し,正しければユーザを作成し,/loginをGETメソッドでルーティングにリクエストする.\\
          正しくなければusers/new.htmlを表示させる.\\
- edit:現在ログインしているユーザの情報を取得して,/users/edit.htmlを表示する. \\
- update:入力された情報が正しい情報かを判断し,正しければユーザ情報を更新し,前回のページを表示させる.

  \item  students\_controller.rb\\
名称:履修者情報処理\\
概要:履修者情報を処理する\\
処理:\\
- new:students/new.htmlを表示させる.\\
- create:入力された情報が正しければ,履修者を作成し,グループでの授業ならば/groupsにGETメソッドでルーティングにリクエストする.\\
          グループの授業でなければ,/homeにGETメソッドでルーティングにリクエストする.\\
          入力された情報が不正ならば,students/new.htmlを表示させる.

  \item  groups\_controller.rb\\
名称:グループ情報処理\\
概要:グループ情報を処理する\\
処理:\\
- index:グループ一覧を取得し,groups/index.htmlを表示させる.\\
- create:入力された情報が正しい情報かを判断し,正しければグループを作成する.その後,/groupsをGETメソッドでルーティングにリクエストする.\\
- update:入力された情報が正しい情報かを判断し,正しければグループ情報を更新し,前回のページを表示させる.

  \item  group\_members\_controller.rb\\
名称:グループメンバ情報処理\\
概要:グルーメンバ情報を処理する\\
処理:\\
- create:入力された情報が正しい情報かを判断し,正しければグループメンバを作成する.その後,/homeをGETメソッドでルーティングにリクエストする.

  \item  questions\_controller.rb\\
名称:質問情報処理\\
概要:質問情報を処理する\\
処理:\\
- index:質問一覧と進捗があれば進捗情報も取得し,/questions/index.htmlを表示する.\\
- past\_index:過去の質問一覧と進捗があれば進捗情報も取得し,/questions/index.htmlを表示する.\\
- new:/questions/new.htmlを表示させる.\\
- create:入力された情報が正しい情報かを判断し,正しければ質問を作成し,/questions/index.htmlを表示させる.\\
          正しくなければ/questions/new.htmlを表示させる.

  \item  progress\_controller.rb\\
名称:進捗情報処理\\
概要:進捗情報を処理する\\
処理:\\
- update:入力された情報が正しい情報かを判断し,正しければ進捗を更新し,/homeをGETメソッドでルーティングにリクエストする.

  \item  achievments\_controller.rb\\
名称:達成情報処理\\
概要:達成情報を処理する\\
処理:\\
- create:入力された情報が正しい情報かを判断し,正しければ達成を作成し,/homeをGETメソッドでルーティングにリクエストする.

  \item  years\_controller.rb\\
名称:開講年度情報処理\\
概要:開講年度情報を処理する\\
処理:\\
- index:現在の授業から開講年度一覧を取得し,/years/index.htmlを表示する.

  \item  lecture\_times\_controller.rb\\
名称:開講回情報処理\\
概要:開講回情報を処理する\\
処理:\\
- index:選択された年度から開講回一覧を取得し,/lecture\_times/index.htmlを表示する.

\end{enumerate}



\newpage

\subsection{Model層}

\begin{enumerate}

\item tb\_user.rb\\
名称:ユーザ情報管理\\
概要:ユーザテーブルの管理を行う\\
関係:\\
- ユーザ 1:N 履修者\\
- ユーザ 1:N 進捗\\
- ユーザ 1:1 公開

\item tb\_student.rb\\
名称:履修者情報管理\\
概要:履修者テーブルの管理を行う\\
関係:\\
- 履修者 N:1 開講年度\\
- 履修者 N:1 ユーザ

\item tb\_group.rb\\
名称:グループ情報管理\\
概要:グループテーブルの管理を行う\\
関係:\\
- グループ 1:N グループメンバ\\
- グループ N:1 開講年度

\item tb\_group\_member.rb\\
名称:グループメンバ情報管理\\
概要:グループメンバテーブルの管理を行う\\
関係:\\
- グループメンバ N:1 グループ\\
- グループメンバ 1:1 履修者\\
- グループメンバ 1:1 進捗

\item tb\_lecture.rb\\
名称:授業情報管理\\
概要:授業テーブルの管理を行う\\
関係:\\
- 授業 1:N 開講年度\\
- 授業 0か1:0か1 公開

\item tb\_lecture\_year.rb\\
名称:開講年度情報管理\\
概要:開講年度テーブルの管理を行う\\
関係:\\
- 開講年度 N:1 授業\\
- 開講年度 1:N 履修者\\
- 開講年度 1:N グループ

\item tb\_lecture\_times.rb\\
名称:開講回情報管理\\
概要:開講回テーブルの管理を行う\\
関係:\\
- 開講回 0か1:0か1 公開\\
- 開講回 1:N 進捗\\
- 開講回 1:N 課題

\item tb\_open\_lecture.rb\\
名称:公開情報管理\\
概要:公開テーブルの管理を行う\\
関係:\\
- 公開 0か1:0か1 ユーザ\\
- 公開 0か1:0か1 開講回

\item tb\_problem.rb\\
名称:課題情報管理\\
概要:課題テーブルの管理を行う\\
関係:\\
- 課題 N:1 開講回\\
- 課題 N:0かN 達成\\
- 課題 N:N 進捗\\
- 課題 1:0かN 質問

\item tb\_progress.rb\\
名称:進捗情報管理\\
概要:進捗テーブルの管理を行う\\
関係:\\
- 進捗 N:1 ユーザ\\
- 進捗 1:1 グループメンバ\\
- 進捗 N:N 課題\\
- 進捗 0かN:0かN 達成

\item tb\_question.rb\\
名称:質問情報管理\\
概要:質問テーブルの管理を行う\\
関係:\\
- 質問 0かN:1 課題

\item tb\_achievment.rb\\
名称:達成情報管理\\
概要:達成テーブルの管理を行う\\
関係:\\
- 達成 0かN:N 課題\\
- 達成 0かN:0かN 進捗

\end{enumerate}
