\section{システム概要}
本システムは、主に授業中の質問や進捗状況をリアルタイムで表示することで、 授業の支援を行うシステムです。
管理者は、本システムが内蔵されている Raspberry Pi 3 を、管理者用の PC に 接続することで利用可能になります。
学生は、この Raspberry Pi 3 に Wi-Fi で接続することで、本システムが利用可能になります。
管理者が担当する授業に合わせて進捗状況の可視化や、質問の蓄積についてを設定できます。

本システムには、ログイン機能、アカウント情報管理機能、授業作成・引き継ぎ機能、グループ情報管理機能、質問・回答機能、
進捗管理機能、課題設定機能があります。

ログイン機能は、管理者や学生が本システムにログインできるようにする機能です。
アカウント情報管理機能は、新規アカウントの登録、アカウントの削除、アカウント登録情報の編集を行います。
授業作成機能は、管理者が質問や進捗の管理を行いたい授業をシステム上に作成する機能です。
授業引き継ぎ機能は、同一の授業を次年度に引き継ぐ場合に設定した課題等を引き継ぐことができる機能です。
グループ情報管理機能は、グループで行う授業について、グループの作成や参加、編集を行うことができる機能です。
質問機能は、学生が教員に対して質問を行うことができる機能です。
回答機能は、学生からの質問に対して管理者が回答することができる機能です。
進捗管理機能は、学生が現在の進捗を入力することで、管理者が学生の進捗をリアルタイムで確認することができる機能です。
課題設定機能は、進捗管理を行いたい課題を設定する機能です。





%このシステムは、主に授業の支援を行うことを目的とするシステムです。(ここ日本語おかしい)
%ず、このシステムを利用するために、アカウントの登録が必要となります。
%初期状態として管理者用のアカウントが登録されており、その管理者用のアカウントからは管理者用のアカウント、
%授業のアシスタントのアカウント、学生のアカウントを作成することができます。
%ログインしていない状態からこのシステムにアクセスすると、ログイン画面が表示されますが、その画面からは
%学生用のアカウントのみ作成することができます。
%このシステムの機能として、匿名の質問システムと進捗管理システムが存在します。

%匿名の質問システムは、現在行われている授業に対して学生用のアカウントから匿名で質問を行い、
%管理者用のアカウントまたはアシスタントのアカウントから回答することができるシステムです。
%学生は他の学生が行った質問とその回答を閲覧することができ、また、過去の授業での質問も閲覧することができます。
%管理者はこれまでに行われた質問に対して編集や削除、公開非公開の設定を行うことができます。

%進捗管理システムは、管理者が授業ごとに課題を設定している場合に利用することができます。(授業の作成の仕方とか課題の設定の仕方を書きたいねこれは)
%管理者が設定した課題の達成状況を学生は入力し、送信することで、管理者がその達成状況をリアルタイムで確認することができます。
%学生は、課題に対する質問や課題の確認依頼を、システムを通して行うことができます。
%また、上記の質問システムと同様、学生は過去の質問閲覧することができます。

%この2種の機能を利用するために、管理者はシステム上に授業を作成する必要があります。
%また、課題が存在する場合には課題の内容も記述し、保存することで進捗管理システムを利用することができるようになります。

%わっかんね〜〜〜むずい 全部の操作とか書いた方がいいんかな〜〜〜




管理者や学生に側をつけるか? 
サブシステム一覧表
