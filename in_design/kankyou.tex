\section{動作環境}

\begin{itemize}

\item 動作環境\\
本システムの動作環境は以下の通りです.
\begin{itemize}
\item raspberry Pi 3 Model B
\item CPU:ARM Cortex-A53
\item GPU:Broadcom VideoCore IV
\item メモリ:LPDDR2 SDRAM 1 GB
\item ストレージ:4 GB eMMC / SDカードPIN
\item OS: Raspbian Stretch
\item webサーバ:Nginx 1.10.3
\item Appサーバ:Ruby on Rails 5.1.4
\item RDBMS:MySQL version 14.14
\end{itemize}

\item 使用ブラウザ
  \begin{itemize}
  \item Google Chrome version 62.0
  \item Firefox version 57.0
  \end{itemize}

\item 開発環境\\
本システムの開発環境は以下の通りです.
  \begin{itemize}
  \item OS:Windows10, ubuntu 16.04 LTS, MacOS Sierra, Raspbian Stretch
  \end{itemize}

\item 使用言語\\
本システムの開発環境は以下の通りです.
\begin{itemize}
\item Ruby on Rails version 5.1.4
\item ruby version 2.4.2
\item HTML5
\item CSS
\item JavaScript
\end{itemize}

\item サーバ:Raspberry Pi 3 Model B

\item データベース:MySQL version 14.14

\item webサーバ:Nginx 1.10.3

\item 文書作成ツール:LaTeX

\item バージョン管理:Git

\item 連絡ツール:Slack

\item 日程管理ツール:Google Calendar

\item UML図作成ツール:Power Point, Excel, A5:SQL Mk-2

\end{itemize}

\section{コード規約}
ファイル命名規則は原則Railsの命名規則に従う

\begin{itemize}
\item コード命名規則
  \begin{itemize}
  \item クラス名,モジュール名は原則UpperCamelCase
  \item 変数名,メソッド名は原則snake\_case
  \item 定数名は全て大文字で'\_'で区切る
\end{itemize}
\item コーディングスタイル
\begin{itemize}
  \item インデントには半角スペース 2 文字を使用する
  \item 文字コードはUTF-8とする
  \item 改行コードにはLFを使用する
  \end{itemize}
\end{itemize}
