\section{サブシステム機能概要一覧}
各サブシステムの機能概要と、その一覧表について記述します。
\subsection{サブシステムごとの機能概要}
各サブシステムごとに、どのような機能の提供を行うのか説明します。
\begin{enumerate}
\item アカウント作成システム(学生用)\\
アカウント作成システム(学生用)では、本システムを利用するために、「ユーザID」、「氏名」、「パスワード」、「パスワードの
確認」を入力し、アカウント新規登録を行う機能を提供します。

\item アカウント作成システム(管理者用)\\
アカウント作成システム(管理者用)では、管理者のアカウントを登録するために、すでに登録されている
別の管理者のアカウントから、他ユーザのアカウントを新規登録することができる機能を提供します。\\
また、登録内容には権限を設定することができ、登録することができる権限は、教員(管理者)、TA、学生となります。

\item アカウント情報編集システム\\
アカウント情報編集システムでは、登録を行ったアカウントに対し、「ユーザID」、「氏名」、「パスワード」を変更
する機能を提供します。\\
なお、アカウント情報を編集するには、「旧パスワード」を入力する必要があります。

\item アカウント削除システム\\
アカウント削除システムでは、アカウントの登録後5年が経過したアカウントに対し、
管理者用ホーム画面から、「アカウントリフレッシュボタン」を押下することで、
アカウントの削除を行う機能を提供します。


\item ログインシステム(学生用)\\
学生アカウントでログインした場合、授業が公開されているときは、その授業の画面が表示されます。\\
また、授業が公開されていないときは、授業非公開の画面が表示されます。

\item ログインシステム(管理者用)\\
ログインシステム(管理者用)では、利用しているユーザを識別するために、「ユーザID」、「パスワード」を入力することで、
登録されているアカウントと照らし合わせ、本システムへのログインを行う機能を提供します。\\
なお、管理者がログインした場合は、管理者用ホーム画面が表示されます。

\item グループ作成システム\\
グループ作成システムでは、グループワークである授業にも対応させるため、
グループ名を入力し、グループを作成する機能を提供します。\\
なお、このグループの作成は学生が行うようになっています。

\item グループ参加システム\\
グループ参加システムでは、作成されたグループへの参加を行うために、
グループを選択した際に、グループの詳細表示および、そのグループへの
参加を行う機能を提供します。

\item グループ編集システム\\
グループ編集システムでは、管理者がグループの作成を行うことや、すでに作成されている
グループおよび、そのグループのメンバを削除する機能を提供します。

\item クイックアクセスシステム\\
クイックアクセスシステムでは、画面右上にクイックアクセスのボタンが存在し、それを展開することで、
「ユーザ氏名」、「ログアウトボタン」、「アカウント編集ボタン」、「現在開講されている講義」が
表示される機能を提供します。

\item 授業作成・編集・引き継ぎシステム\\
授業作成・編集・引き継ぎシステムでは、管理者が本システムを利用したい授業に合わせて、
「授業名」、「授業形態」、「授業回」、「課題数」、を入力し、授業の作成を行うための機能
を提供します。\\
また、「編集ボタン」を押下することで、すでに作成した授業の編集を行うことも可能となります。\\
「引継ボタン」を押下することで、過去の授業の設定内容を引き継いで、新しく授業を作成することができます。
このとき、引き継がれた授業の質問情報は過去のものとして引き継がれます。

\item 課題作成システム\\
課題作成システムでは、授業の作成・編集・引き継ぎ時に、授業回を選択するとで、
課題名および課題の内容を設定できる機能を提供します。

\item 授業公開・非公開システム\\
授業公開・非公開システムでは、授業中に本システムを利用する際に、
授業回を選択することで、その選択した授業を学生に公開する機能を提供します。
また、クイックアクセスから非公開を選択することで、授業を非公開にし、終了することができます。

\item 進捗確認システム\\
進捗確認システムでは、授業作成時に設定した課題に応じて、学生の課題の進捗状況を
確認することができる機能を提供します。\\
なお、管理者の進捗確認画面には、学生の課題に対する進捗状況が表示されており、
この画面はグループワークであるか否かによって変わります。\\
また、学生の状態を確認するための「状態」という欄があり、課題に対して学生からの質問があった場合は「質問マーク」、
課題の達成確認をしてほしい場合は「確認マーク」、PCが動かなくなったなどの緊急を要する場合は「緊急マーク」が表示されます。

\item 進捗送信システム\\
進捗送信システムでは、学生が課題を達成した際に、その課題にチェックを行い、
「更新ボタン」を押下することで、管理者の進捗確認画面へと進捗状況が反映される機能を提供します。

\item 質問閲覧システム\\
質問閲覧システムでは、授業中に学生が送信した質問を閲覧することができる機能を提供します。

\item 過去の質問閲覧システム\\
過去の質問閲覧システムでは、現在開講している授業の過去に得られた質問を閲覧したい場合に、
「年度」と「授業回」を選択していくことで、過去の質問を閲覧することができる機能を提供します。

\item 質問送信システム\\
質問送信システムでは、「質問内容」を入力し、「質問ボタン」を押下することで、
学生が授業中に管理者へ質問を送信することができる機能を提供します。

\item 質問回答システム\\
質問回答システムでは、管理者が授業中に学生から送信された質問に対して、回答内容を入力し、
「回答ボタン」を押下することで、学生に回答を送信する機能を提供します。\\
また、回答に関しては、過去の質問閲覧に回答内容を公開するかの設定が可能となります。

\item 質問編集システム\\
質問編集システムでは、「編集ボタン」を押下することで、管理者が質問や回答の内容や、
質問を公開するかの設定について、編集することができる機能を提供します。

\item 質問削除システム\\
質問削除システムでは、「削除ボタン」を押下することで、
保存されている質問と回答を削除することができる機能を提供します。

\end{enumerate}

\subsection{サブシステム一覧表}
各サブシステムで使用されるデータベースと、行われる処理のフロー番号を表1に示します。\\
なお、処理フロー番号は5章のフローチャートと対応しています。
\begin{table}[h]
		\centering
		\caption{サブシステム一覧}
		\label{サブシステム一覧}
		\begin{tabular}{|l|l|l|}
		\hline
		サブシステム & 使用データベース  & 処理フロー番号\\ \hline\hline
		アカウント作成システム(学生用) & ユーザテーブル & (1),(2),(3)\\ \hline
		アカウント作成システム(管理者用) & ユーザテーブル & (4),(5),(6)\\ \hline
		アカウント情報編集システム & ユーザテーブル & (6),(7),(8)\\ \hline
		アカウント削除システム & ユーザテーブル & (9),(10)\\ \cline{2-2}
                        & 履修者テーブル & \\ \cline{2-2}
                        & グループテーブル & \\ \cline{2-2}
                        & グループメンバテーブル & \\ \hline
		ログインシステム(学生用) & ユーザテーブル & (11),(12),(13),(14),(15)\\ \cline{2-2}
		                        & 履修者テーブル & (16),(17),(18),(19),(20)\\ \cline{2-2}
                            & 公開テーブル & (21),(22),(23),(24),(25)\\ \hline
    ログインシステム(管理者用) & ユーザテーブル & (26),(12)\\ \cline{2-2}
		                        & 開講年度テーブル & \\ \cline{2-2}
                            & 授業テーブル & \\ \hline
    グループ作成システム & グループテーブル & (27),(28),(29)\\ \hline
    グループ参加システム & グループテーブル & (30),(21),(31),(32)\\ \cline{2-2}
                      & グループメンバテーブル & \\ \hline
    グループ編集システム & グループテーブル & (30),(21),(33),(34),(35)\\ \cline{2-2}
                      & グループメンバテーブル & (36),(37)\\ \hline
    クイックアクセスシステム & 開講回テーブル & (38),(14)\\ \cline{2-2}
                         & 授業テーブル & \\ \cline{2-2}
                         & 公開テーブル & \\ \hline
    授業作成・編集・引き継ぎシステム & 開講年度テーブル & (39),(16),(40),(41),(42)\\ \cline{2-2}
                                & 開講回テーブル & (43),(44),(45)\\ \cline{2-2}
                                & 授業テーブル & \\ \cline{2-2}
                                & 公開テーブル & \\ \cline{2-2}
                                & 課題テーブル & \\ \hline
                              \end{tabular}
                          	\end{table}

\clearpage
                            \begin{table}[h]
                                \centering
                                \begin{tabular}{|l|l|l|}
                                \hline
    課題作成システム & 開講回テーブル & (46),(47),(48),(43)\\ \cline{2-2}
                  & 課題テーブル & \\ \hline
    授業公開・非公開システム & 開講年度テーブル & (49),(50),(51),(52),(38)\\ \cline{2-2}
                        & 開講回テーブル & (53),(54),(55),(56)\\ \cline{2-2}
                        & 授業テーブル & \\ \cline{2-2}
                        & 公開テーブル & \\ \cline{2-2}
                        & 課題テーブル & \\ \cline{2-2}
                        & 進捗テーブル & \\ \hline
    進捗確認システム & ユーザテーブル & (50),(52),(57),(58),(59)\\ \cline{2-2}
                  & グループテーブル & (19)\\ \cline{2-2}
                  & 開講回テーブル & \\ \cline{2-2}
                  & 課題テーブル & \\ \cline{2-2}
                  & 進捗テーブル & \\ \cline{2-2}
                  & 達成テーブル & \\ \hline
    進捗送信システム & 達成テーブル & (60),(61)\\ \hline
    質問閲覧システム & ユーザテーブル & (11),(13),(17),(20),(22)\\ \cline{2-2}
                  & 課題テーブル & (62),(63),(64),(18)\\ \cline{2-2}
                  & 質問テーブル & \\ \hline
    過去の質問閲覧システム & ユーザテーブル & (65),(15),(66),(43),(67)\\ \cline{2-2}
                        & 開講年度テーブル & (63),(18)\\ \cline{2-2}
                        & 開講回テーブル  & \\ \cline{2-2}
                        & 課題テーブル & \\ \cline{2-2}
                        & 質問テーブル & \\ \hline
    質問送信システム & ユーザテーブル & (68),(69),(70),(71)\\ \cline{2-2}
                  & 課題テーブル & \\ \cline{2-2}
                  & 質問テーブル & \\ \hline
    質問回答システム & ユーザテーブル & (72),(63),(18),(73),(74)\\ \cline{2-2}
                  & 課題テーブル & \\ \cline{2-2}
                  & 質問テーブル & \\ \hline
    質問編集システム & 質問テーブル & (75),(15),(76),(43),(77)\\ \cline{2-2}
                  &            & (63),(18),(78),(74)\\ \hline
    質問削除システム & 質問テーブル & (75),(76),(77),(79),(80)\\ \hline
		\end{tabular}
	\end{table}
